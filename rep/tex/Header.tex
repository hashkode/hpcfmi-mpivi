%%%%%%%%%%%%%%%%%%%%%%%%%%%%%%%%%%%%%%%%%%%%%%%%%%%%%%
%% Technische Universität München
%% Lehrstuhl für Elektrische Energiespeichertechnik
%% Latex-Vorlage für Short Paper/Hauptseminar
%% Kontakt: 
%% Stand: 10.05.2013
%%%%%%%%%%%%%%%%%%%%%%%%%%%%%%%%%%%%%%%%%%%%%%%%%%%%%%

%% *** IEEE (S) Dokumentklasse ***
\documentclass[10pt]{IEEEStran}


%% *** Schriften und Sprache ***
	\usepackage[utf8]{inputenc}		% Texcodierung
	\usepackage[T1]{fontenc}				% T1 Schriften
	\usepackage[german]{babel}			% deutsche Silbentrennung und Beschriftungen
	%\usepackage{mathptmx} 					% Times Schriftart
	%\usepackage[scaled]{helvet} 		% Helvetica Serifenfreie Schriftart
	%\usepackage{courier}
	\usepackage[babel,german=quotes]{csquotes} % Deutsche Anf�hrungszeichen mittels \enquote{}
	\usepackage{xspace}							% Intelligente Leerzeichen


%% *** Bilder und Geometrie ***
	\usepackage[pdftex]{graphicx}			% Grafiken einbinden
	
		\graphicspath{{img/}} 												% Pfad f�r Bilder angeben
		\DeclareGraphicsExtensions{.pdf,.jpeg,.png}		% Endungen die eingebunden werden
	\usepackage[caption=false,font=footnotesize]{subfig}	% Caption wird von IEEE �bernommen: http://www.ctan.org/tex-archive/macros/latex/contrib/subfig/
	\usepackage[a4paper,left=1.65cm,right=1.65cm,top=1.78cm,bottom=1.78cm,includeheadfoot]{geometry}
	%\usepackage{pdfpages}						% PDF Dateien einf�gen
	%\usepackage{color}								% Farben erm�glichen
	%\usepackage{capt-of}
	%\usepackage{setspace}	


%% *** Tabellen ***
	%\usepackage{longtable}						% Tabellen �ber mehrere Seiten
	\usepackage{array}								% erweitern Tabelleneigenschaften: http://www.ctan.org/tex-archive/macros/latex/required/tools/
	\usepackage{mdwtab}								% http://www.ctan.org/tex-archive/macros/latex/contrib/mdwtools/
	%\usepackage{booktabs}
	%\usepackage{tabularx}
	%\newcolumntype{C}{>{\centering\arraybackslash}X}


%% *** Mathe ***
	\usepackage[cmex10]{amsmath}
		\interdisplaylinepenalty=2500 	% Verhindert Seitenumbruch bei l�ngeren Formeln
	\usepackage{mdwmath}							% http://www.ctan.org/tex-archive/macros/latex/contrib/mdwtools/
	\usepackage{amsmath,amsfonts,amssymb}
	\usepackage[squaren,textstyle]{SIunits}
	%\usepackage{siunitx}  
	%\sisetup{locale = DE, per-mode=symbol,range-units=single} 
	\usepackage{icomma}

%Abkürzungsverzeichnispakete
\usepackage[nolist]{acronym}
	
	
	
%% *** Algorithmen ***
	%\usepackage{algorithmic} 				% http://www.ctan.org/tex-archive/macros/latex/contrib/algorithmicx/
		%\floatname{algorithm}{Algorithmus}
		%\renewcommand{\algorithmicfor}{\textbf{F�r}}
		%\renewcommand{\algorithmicdo}{\textbf{:}}
		%\renewcommand{\algorithmicendfor}{}

	
%% *** Quellcode ***
	%\usepackage{listings}						% Direktes einbinden von Quellcode
	%\usepackage{verbatim}						% Quellcode anstatt Listings
	%\usepackage[numbered]{mcode}			% MatLab Code
	%\renewcommand{\lstlistingname}{Quellcode}


%% *** Einstellungen ***
	%\setcounter{secnumdepth}{4}				% Kapitelnummerierung mit x Ebenen
	%\setcounter{tocdepth}{4}					% Eintrag ins Inhaltsverzeichnis bis Ebene x
	%\definecolor{Gray}{gray}{0.4}
	%\onehalfspacing
		
	%\setlength{\parindent}{0mm}			% Erstzeileneinzug bei neuem Absatz
	%\setlength{\parskip}{8pt}				% Abstand zwischen Abs�tzen einstellen
	%\setcapindent{1em}								% Zeilenumbruch bei Bildbeschreibungen


%% *** Kopf- und Fusszeile ***
	%\usepackage{scrpage2}							% Seitenlayout KOMA-Script
	%\pagestyle{scrheadings}
	%
	%\automark[section]{chapter}
	%\setheadsepline{1.0pt}[\color{Gray}]
	%
	%\ihead{}
	%\chead{}
	%\ohead{\color{Gray}\textbf{\headmark}}

	
%% *** Nummerierung und Beschriftung ***
	%\addtokomafont{captionlabel}{\normalfont\bfseries}
	%\addtokomafont{caption}{\normalfont\bfseries}


%% *** Literatur ***
	%\usepackage{cite}								% http://www.ctan.org/tex-archive/macros/latex/contrib/cite/
	% cite.sty was written by Donald Arseneau
	% V1.6 and later of IEEEtran pre-defines the format of the cite.sty package
	% \cite{} output to follow that of IEEE. Loading the cite package will
	% result in citation numbers being automatically sorted and properly
	% "compressed/ranged". e.g., [1], [9], [2], [7], [5], [6] without using
	% cite.sty will become [1], [2], [5]--[7], [9] using cite.sty. cite.sty's
	% \cite will automatically add leading space, if needed. Use cite.sty's
	% noadjust option (cite.sty V3.8 and later) if you want to turn this off.
	% cite.sty is already installed on most LaTeX systems. Be sure and use
	% version 4.0 (2003-05-27) and later if using hyperref.sty. cite.sty does
	% not currently provide for hyperlinked citations.
	
	\usepackage[numbers]{natbib}
	
	% *** Chronologische Sortierung ***
	%\bibliographystyle{EES_DIN1505_UNSRT}
	
	% *** Alphabetische Sortierung ***
	%\bibliographystyle{EES_DIN1505_ABBRV}
	\bibliographystyle{EES_STANDARD}
	
	\renewcommand{\refname}{List of Bibliography}	% Literatur in Literaturverzeichnis umbenennen


%% *** PDF, URL und HYPERLINK Packete ***
%% *** MEISTENS ALS LETZES EINBINDEN! ***
	\usepackage{url}															% Einf�gen von URLs: \url{my_url_here}
	\usepackage{float}														% http://www.ctan.org/tex-archive/macros/latex/contrib/float
	\newcommand\MYhyperrefoptions{
		bookmarks					= true,
		bookmarksnumbered	= true,
		pdfpagemode				= {UseOutlines},
		plainpages				= false,
		pdfpagelabels			= true,
		colorlinks				= true,
		linkcolor					= {black},
		citecolor					=	{black},
		urlcolor					= {blue},
		bookmarksopen			= false
	}

	\ifCLASSINFOpdf
		\usepackage[\MYhyperrefoptions,pdftex]{hyperref}
	\else
		\usepackage[\MYhyperrefoptions,breaklinks=true,dvips]{hyperref}
		\usepackage{breakurl}
	\fi
	
	\hypersetup{
		pdftitle			= {Hauptseminar},
		pdfauthor			= {\AutorVorname \AutorNachname},
		pdfsubject		= {\Thema},
		pdfkeywords		= {\Keywords},
		pdfcreator		= {LaTeX},
		pdfproducer		= {LaTeX}
	}
