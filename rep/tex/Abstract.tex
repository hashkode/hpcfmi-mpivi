\begin{abstract}
\label{sec:abstract}
Verloren in Raum und Zeit? Nicht mehr! Für alle die regelmäßig eine Ausfahrt auf dem Weg von Terra nach Alpha Centauri verpassen und unterwegs mit leerem Tank auf einem leeren Planeten landen, haben wir eine optimale Lösung entwickelt: skalierbare asynchrone Value Iteration per Open MPI.
Ziel dieser Ausarbeitung ist die Einführung in die relevanten Hintergründe zu Open MPI und darauf aufbauend die Motivation eines Projektaufbaus, der die Beurteilung verschiedener Kommunikationsschemata und Parametrierungen erlaubt. Mittels dieses Frameworks können wir aus drei MPI Schemata, sechs Ausführungsumgebungen und diversen Parameterkombinationen je nach Größe des Problems und zur Verfügung stehender Rechenumgebung eine zielführende Kombination ableiten.
Die Kernergebnisse sind die Identifikation verschiedener Zusammenhänge zwischen MPI Kommunikationsschema, Rechenumgebung und Parametrierung und Qualitätsmetriken wie Rechenzeit, Speicherbedarf und Lösungsqualität. Diese erlauben eine optimale Anpassung des Projekts an die jeweiligen Rahmenbedingungen.

\end{abstract}