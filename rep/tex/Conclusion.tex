\section{Erkenntnisse}
\label{sec:erkenntnisse}
Im Rahmen dieser Arbeit konnten wir zeigen, dass automatisiertes Ausführen und Testen des vorliegenden Optimierungsproblems
auf unterschiedlichen Targets mit unterschiedlichen Parametern realisierbar ist.
Unterschiede zwischen verschiedenen MPI-Schemata und deren Einflüsse auf Performance und Messwerte der Value Iteration wurden dargelegt.
Weiterhin wurde der Einfluss verschiedener Hardware-Strukturen sowie Parameter auf die Value Iteration dargestellt.

Wir konnten zeigen, dass der benötigte Arbeitsspeicher pro Processor vom verwendeten Schema und dessen Implementierung abhängt,
die benötigte Zeitdauer hängt beim vorliegenden Problem in erster Linie von der Größe des com\_intervals und der gewählten Hardware ab.

Es konnte kein eindeutiger und starker Zusammenhang zwischen runtime und world\_size festgestellt werden was zu dem Schluss führt, dass
die Lösung des vorliegenden Value Iteration Problems mit MPI keinen signifikanten Mehrwert mit sich bringt.